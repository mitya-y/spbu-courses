% !TeX spellcheck = ru_RU
% !TEX root = vkr.tex

\section*{Введение}
\thispagestyle{withCompileDate}

В последние годы графовые базы данных набирают все большую популярность:
они используются в алгоритмах для социальных сетей~\cite{SocNets}, в создании баз 
знаний~\cite{Knowledge}, в медицине~\cite{ChemRecEngine}. Хотя они и имеют ряд ограничений
и недостатков, во многих задачах они показывают себя намного лучше, чем реляционные базы
данных~\cite{GraphDatabases}.
В них данные представляются как граф с метками на вершинах (которые представляют объекты)
и на ребрах (которые представляют отношения между объектами). Одна из естественных задач,
возникающая при работе с такими базами данных, --- нахождение всех объектов, связанными
определенными отношениями с заданным. Для решения ее надо искать в графе пути с
определенными ограничениями. Если рассматривать всевозможные метки на ребрах как алфавит,
а путь в графе как слова, то ограничения можно проклассифицировать формальными языками~\cite{FormPathReg}.
Бyдем рассматривать регулярные языки над этим алфавитом и использовать регулярные ограничения,
а точнее их расширение –-- двухсторонние регулярные ограничения, позволяющие
ходить по ребрам в обратном направлении. Запросы с двухсторонними регулярными
ограничениями были реализованы в языке запросов
к данным SPARQL, и добавлены в ISO стандарт языков запросов к графам в ISO 
39075:2024\footnote{Стандарт ISO/IEC 39075:2024: \url{https://www.iso.org/standard/76120.html} (Дата посещения: 09.11.2024)}.

В реализациях алгоритмов для решения этой задачи активно используются представление графов в
виде разреженных матриц смежности, например в графовых базах данных MilleniumDB~\cite{MilleniumDB} и FalcorDB\footnote{Документация FalcorDB: \url{https://docs.falkordb.com/} (Дата посещения: 08.01.2025)}. 
Так же есть алгоритмы, представленные в работax~\cite{sparse-rpq-1} и~\cite{OldRpqVkr}, основанные на операциях линейной алгебры.

Операции линейной алгебры отлично ложатся на огромную вычислительную мощность видеокарт,
и хочется использовать этот потенциал для улучшения времени работы алгоритма. Есть
различные библиотеки, позволяющие работать с булевой линейной алгеброй на 
GPU: Spla~\cite{Spla}, Cusp\footnote{Репозитория Cusp на GitHub: \url{https://github.com/cusplibrary/cusplibrary} (Дата посещения: 08.01.2025)}, cuBool~\cite{CuBool}.
Библиотека cuBool~\cite{CuBool} в тестах\footnote{Тесты производительности cuBool: \url{
https://github.com/SparseLinearAlgebra/cuBool?tab=readme-ov-file\#performance} (Дата посещения: 08.01.2025)}
производительности показывает себя лучше всего, но она не поддерживает последнюю
версию вычислительного API CUDA и в ней нет некоторой функциональности,
необходимой для реализации алгоритма.

В данной работе будет представлена адаптация алгоритма достижимости с регулярными 
ограничениями для GPGPU вычислений и его 
реализация на вычислительном API CUDA для графических ускорителей.