% !TeX spellcheck = ru_RU
% !TEX root = vkr.tex

\section*{Введение}
\thispagestyle{withCompileDate}

В последние годы графовые базы данных набирают все большую популярность:
они используются для анализа социальных сетей~\cite{SocNets}, в создании баз 
знаний~\cite{Knowledge}, в медицине~\cite{ChemRecEngine}. Хотя они и имеют ряд ограничений
и недостатков, во многих задачах они показывают себя намного лучше, чем реляционные базы
данных~\cite{GraphDatabases}.
В них данные представляются как граф с метками на вершинах (которые представляют объекты)
и на ребрах (которые представляют отношения между объектами). Одна из естественных задач,
возникающая при работе с такими базами данных, --- нахождение всех объектов, связанными
определенными отношениями с заданным. Для решения ее надо искать в графе пути с
определенными ограничениями. Если рассматривать всевозможные метки на ребрах как алфавит,
а путь в графе как слова, то ограничения можно проклассифицировать формальными языками~\cite{FormPathReg}.
Бyдем рассматривать регулярные языки над этим алфавитом и использовать регулярные ограничения,
а точнее их расширение –-- двухсторонние регулярные ограничения, позволяющие
ходить по ребрам в обратном направлении. Запросы с двухсторонними регулярными
ограничениями были реализованы в языке запросов
к данным SPARQL, и добавлены в ISO стандарт языков запросов к графам в ISO 
39075:2024\footnote{Стандарт ISO/IEC 39075:2024: \url{https://www.iso.org/standard/76120.html} (Дата доступа: 19.06.2025)}.

В реализациях алгоритмов для решения этой задачи активно используются представление графов в
виде разреженных матриц смежности, например в графовых базах данных MilleniumDB~\cite{MilleniumDB} и FalcorDB\footnote{Документация FalcorDB: \url{https://docs.falkordb.com/} (Дата доступа: 19.06.2025)}. 
Так же есть алгоритмы, представленные в работax~\cite{sparse-rpq-1} и~\cite{OldRpqVkr}, основанные на операциях линейной алгебры.

Операции линейной алгебры отлично ложатся на огромную вычислительную мощность видеокарт,
и хочется использовать этот потенциал для улучшения времени работы алгоритма. 

В предыдущей работе~\cite{PrevWork} был реализован алгоритм~\cite{OldRpqVkr} для вычислений на GPU c помощью библиотеки cuBool~\cite{CuBool}. Тестирование на наборе данных Wikidatda~\cite{wikidata} показало, что реализация на GPU\footnote{\href{https://github.com/mitya-y/rpq/tree/cb2583e64e51f28dfe16f8e2b66732bc44bb04d5}{Код} реализации алгоритма на GPU, представленный в предыдущей работе (Дата доступа: 19.06.2025)} отработала в 4 раза хуже реализации на CPU\footnote{\href{https://github.com/SparseLinearAlgebra/LAGraph/tree/2-rpq}{Код} реализации алгоритма на CPU (Дата доступа: 19.06.2025)}. 

В данной работе будут изучены причины, по которым реализация на GPU отставала по производительности от CPU и предложены изменения реализации на GPU, улучшающие ее производительность. 

